\subsection{Objetivos generales}

ICMP es el módulo del protocolo TCP/IP que se encarga de proveer mensajes de error y de control cuando
se produce alguna anormalidad en el envío de un datagrama IP. Existen distintas herramientas
que se apoyan sobre este módulo para obtener información acerca del estado de las rutas que
debe atravesar un paquete para llegar al destino, como por ejemplo, el valor del RTT (round-trip-time)
entre 2 hosts. Entre estas herramientas se destacan los comandos \emph{ping} y \emph{traceroute}.

El objetivo de este trab\'ajo pr\'actico consiste en realizar nuestro propio
\emph{traceroute} basado en Scapy y realizar distintos experimentos con él.
Esta herramienta es utilizada principalmente para hacer un an\'alisis estad\'istico de tiempos
de RTT en la comunicaci\'on entre hosts. Con esto buscamos comprender el funcionamiento del protocolo
ICMP: \emph{Internet control message protocol}, que actúa sobre la capa de red.

Además de presentar un desafío implementativo que genera aportes a nivel teórico, nos interesa
observar el protocolo funcionando en un contexto real. Por lo tanto uno de los aspectos
más relevantes en el tp es el análisis de la topología y distribución geográfica del conjunto de enlaces
y routers atravesados por los paquetes ICMP enviados entre los distintos hosts. En suma a lo
anteriormente mencionado queremos determinar la presencia o ausencia de enlaces submarinos en
la ruta de los paquetes, donde el RTT de los mismos se destaque por encima de los
demás debido a las largas distancias.

Por todos estos motivos, los hosts destino de los paquetes fueron seleccionados entre un conjunto de universidades ubicadas en distintos países y continentes. De esta manera generamos casos de
prueba en donde los paquetes tengan que atravesar gran cantidad de tramos,
aumentando las probabilidades de que transiten por enlaces de diversos tipos, incluyendo submarinos.


\subsection{Tipos de mensajes ICMP}

Los mensajes de ICMP se transmiten en forma de datagramas. El emisor puede ser tanto un host como router,
y el destino es siempre la dirección source del datagrama IP que motivó el mensaje.

Entre los tipos de mensajes de error más comunes se destacan:
\begin{itemize}
  \item Echo reply: Respuesta a echo request. Los datos recibidos por el request deben ser incluídos
  en el mensaje. type: 0
  \item Destination host unreachable: El router no encuentra en su tabla una dirección a la cual
  forwardear el paquete. type: 3
  \item Redirect: El router que recibe el paquete detecta que otro router ofrece un camino más
  efectivo para forwardear el paquete. type: 5
  \item Echo request: Se espera que los datos enviados sean recibidos nuevamente en un mensaje echo
  reply. type: 8
  \item Time exceeded: Mensaje generado por un router para indicarle al host emisor de un datagrama
  que su TTL (''Time to live'') ha alcanzado el valor 0.

\end{itemize}


\subsection{Ping}

El comando ping es una tool que permite testear la actividad de un host dentro del protocolo IP y medir
el RTT entre el dispositivo que ha ejecutado el comando y el host de interés. El mismo se basa en
la emisión de mensajes \emph{echo request}, y sus respectivas respuestas \emph{echo reply}. El RTT queda
determinado entonces por el tiempo que tarda el host emisor en recibir la respuesta.
Los requests para los cuales no se recibe ninguna respuesta son registrados como paquetes perdidos.

\subsection{Traceroute}

A diferencia de ping, \emph{traceroute} provee información acerca de la ruta que siguen los datagramas
para arribar al destino. Esta herramienta arma un registro ordenado con los routers que ha atravesado
el paquete, junto con el valor de los RTT entre cada par sucesivo de dichos routers. La suma de todos
esos tiempos indica el RTT entre el nodo emisor y el destino. En la siguiente sección explicaremos el
mecanismo de esta herramienta con mucho mayor detalle.
