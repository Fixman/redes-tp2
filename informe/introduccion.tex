El objetivo de este trab\'ajo pr\'actico consiste en realizar nuestro propio 
Traceroute como herramienta para hacer un an\'alisis estad\'istico de tiempos
de RTT en la comunicaci\'on con algunos hosts ubicados en distintos lugares
geográficos (universidades en distintos pa\'ises).

El fin de este análisis, además de presentar un desafio implementativo y analítico, sería determinar haciendo uso de esta herramienta enlaces submarinos donde el RTT de los mismos se destaque por encima de los demas debido a las largas distancias.

Más especificamente, \textcolor{red}{estimamos un RTT promedio} para cada salto que se da en 
la ruta que nos comunica con cada host, y \textcolor{red}{normalizamos este valor para 
conseguir un ZRTT} para cada salto que nos de suficiente informaci\'on como para
\textcolor{red}{analizar el trasfondo del tr\'afico en la red}.



